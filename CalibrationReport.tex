%%=======================================================================
%% This shows many features of MSLCalCert class 
%% for producing calibration reports - Blair Hall, March 2020							%%=======================================================================

% Class options: CIPM, IANZ, unendorsed (default: CIPM) - note, case sensitive
\documentclass[IANZ]{MSLCalCert} 

\begin{document}
%\raggedright	
%\onehalfspacing

% Cover page details
 \date{19 February 2018}   % Can use \today until the report date is known
 
 \reportnumber{Electrical/2018/Sxxx}
 \fileref{J01702}
 
 \title{
 	(A fictitious) report on the calibration of an type-N male open, 
 	serial number: 54673
 }
 \maketitlepage

	
% Possible section headings: Description, Identification, Client
% Reference, Date(s) of Calibration (of Test), Objective
% Method, Conditions, Notes, Results, Uncertainty, Conclusion
%
\section{Description}
The component is from a Keysight vector network analyser calibration kit model 85032F. 


\section{Identification}
The component serial number is 54673.

\section{Client}
Airways Corporation of New Zealand Ltd, 50 Tacy St, 6022 Kilbirnie, Wellington.

\section{Date of Calibration}
The measurements were performed on the 7$^\mathrm{th}$ of February 2018.

\section{Conditions}
Ambient temperature was maintained within $\SI{\pm 1}{\celsius}$ of $\SI{23}{\celsius}$.

\section{Method}
Measurements of the voltage reflection coefficient were made according to procedure MSLT.E.063.002. 

\section{Results}
Results are reported in polar coordinates (magnitude, $\rho$, and phase, $\phi$), using a linear scale for magnitude and units of degrees for phase. 

% Although not used here, it is also possible to have a \subsubsection{}
\subsection{Open (male), SN 54673}

 \begin{center} % Centered horizontally on the page
 
 % The report text has 1.5 line spacing, 
 % but that is too wide for tables 
 \begin{singlespace}
 
 	\small	% use a smaller font size for the table entries
 
  	% Increases the vertical spacing between rows slightly  
  	\setlength{\extrarowheight}{3pt}
  
	\[
		% the 'S' array column type will align numbers on the decimal 
        % Note 'S[group-minimum-digits=3]' or '\sisetup{group-minimum-digits=3 }'
        % should be used to force a space separator every 3 digits (this
        % does not happen by default until there more than 4 digits)
  		\begin{array}{SSSSS}
    		\multicolumn{1}{c}{ \text{frequency} } & 
    		\multicolumn{2}{c}{ \text{magnitude} } &
    		\multicolumn{2}{c}{ \text{phase} } 
    		\\
		% 2nd line 
    		\multicolumn{1}{c}{ \si{(MHz)} } &  
    		\multicolumn{2}{c}{ (\text{linear}) } &
    		\multicolumn{2}{c}{ \text{(/degree)} }  
    		\\
  		% 3rd line 
     		& {\rho} & {U(\rho)} & {\phi} & {U(\phi)} 
     		\\ \hline % Underline the headings

  		%%-----------------------------------------------
  		% Data here
		45 &   0.9998 &   0.0023$^\dagger$ &    -1.46 &     0.13     \\
		50 &   0.9998 &   0.0023$^\dagger$ &    -1.62 &     0.13     \\
		100 &   0.9999 &   0.0023$^\dagger$ &    -3.27 &     0.13    \\
		300 &   0.9998 &   0.0025 &    -9.80 &     0.14    \\
		500 &   0.9997 &   0.0026 &   -16.34 &     0.15    \\
		1000 &   1.0000 &   0.0032 &   -32.72 &     0.18   \\
		2000 &   0.9994 &   0.0054 &   -65.67 &     0.31  \\
		3000 &    1.000 &    0.011 &   -98.66 &     0.62   \\
		4000 &    0.999 &    0.013 &  -131.74 &     0.78   \\
		5000 &    0.999 &    0.016 &  -164.77 &     0.90   \\
		6000 &    0.998 &    0.017 &  +162.15 &     0.99   \\
		7000 &    0.997 &    0.018 &   +129.0 &      1.1   \\
		8000 &    0.997 &    0.018 &    +95.9 &      1.1   \\
		9000 &    0.996 &    0.018 &    +62.7 &      1.1  \\
		%%-----------------------------------------------
		
		\end{array}
	\]
	
\end{singlespace}
\end{center}


\section{Uncertainty}
A coverage factor $k=1.96$ was used to calculate the expanded uncertainties $U(\cdot)$ at a level of confidence of approximately 95\%. The number of degrees of freedom associated with each measurement result was large enough to justify this coverage factor.  

Some of the expanded uncertainty values reported fall outside MSL's current IANZ scope of accreditation. These values are indicated by a $\dagger$. The least expanded uncertainty for a magnitude measurement close to unity in the MSL scope is currently 0.0024. 

% Use breaks to fine-tune the layout when the report is complete
\pagebreak

% A \paragraph is a lower heirarchy section. The 'heading' text will be in bold
% and the 'body' text will follow on the same line.
\paragraph{Note:} \referenceGUM	% Standard reference to the GUM

%==============================================================
% Signatures
% \clearpage % This will start a new page for the signatures

% Who will sign? 
% Enter names here (up to '\signatureE', if needs be)
% For text under the name use '\signatureARole{}', etc
\signatureA{Blair Hall} 
\signatureB{Keith Jones} \signatureBRole{Distinguished Scientist}

% Choose between the two roles:
%\chiefMetrologist{
\chiefMetrologistDelegate{
Mark Clarkson
}%

% Append letters (up to '\signaturesABCDE') to match those in
% the '\signature' definitions above.
\signaturesAB

\end{document}
